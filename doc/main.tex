\chapter{Digtalisierten Dokumente}

% Warum scannt man Dokumente

% Was für Dokumente werden gescannt

% 
Schon in den 50er Jahren begann die Forschung im Bereich der Optischen Zeichenerkennung (engl. OCR)\autocite{BairdEvolutionDocumentImage2014}. OCR fand zuerst Anwendung in genau spezifizierten Problembereichen zum Beispiel die Erkennung von Druckbuchstaben einer Schreibmaschine. 
Je mehr Dokumente digitalisiert wurden, desto klare wurde es das Dokumente mehr als 
eine Kette von Zeichen sind. 

\chapter{Selbstüberwachtes Lernen}

% \begin{figure}
%     \includegraphics[]{figures/graphs/exp.pdf} 
%     \caption{Image}
%   \end{figure}

\chapter{Umsetzung}

\section{Evaluierung}

% \csvautotabular{results/DIVA-HisDB.csv}

\newcommand{\resulttable}[2]{
    \begin{tabular}{l|r|r|r|r|r}%
        \hline
        \csvreader[head to column names, filter equal={\dataset}{#1}]{results/document_image_segmentation_results.csv}{}%
        {#2}
        \\\hline
        \end
        {tabular}
}

\resulttable{CB55}{\\ \name & \pixelacc & \FgPA & \meanacc & \meanIU & \fwIU}
\resulttable{CSG18}{\\  \pixelacc & \FgPA & \meanacc & \meanIU & \fwIU}
\resulttable{CSG863}{\\  \pixelacc & \FgPA & \meanacc & \meanIU & \fwIU}
