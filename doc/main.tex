\chapter{Digtalisierten Dokumente}

% Warum scannt man Dokumente

% Was für Dokumente werden gescannt

% 
Schon in den 50er Jahren begann die Forschung im Bereich der Optischen Zeichenerkennung (engl. OCR)\autocite{BairdEvolutionDocumentImage2014}. OCR fand zuerst Anwendung in genau spezifizierten Problembereichen zum Beispiel die Erkennung von Druckbuchstaben einer Schreibmaschine. 
Je mehr Dokumente digitalisiert wurden, desto klare wurde es das Dokumente mehr als 
eine Kette von Zeichen sind. 

\section{Schritte in der Verarbeitung von Dokumentenbildern}
Dokumentensegmetierung
OCR
Klassifizierung siehe \cite{McConnaugheyLabeledSegmentationPrinted2017}
Flow
\section{Dokumentensegmetierung mittels CNN}
1. Ansatz von \autocite{ChenConvolutionalNeuralNetworks2017}
Superpixel segmentierung
Klassifizierung auf Superpixel statt Pixelebene.

2. Ansatz von \autocite{WickFullyConvolutionalNeural2017}

\section{SLIC Superpixel}
\cite{AchantaSLICSuperpixels2010}

\chapter{Selbstüberwachtes Lernen}

Jigsaw
\cite{NorooziUnsupervisedLearningVisual2016}
% \begin{figure}
%     \includegraphics[]{figures/graphs/exp.pdf} 
%     \caption{Image}
%   \end{figure}

\chapter{Umsetzung}

\section{Evaluierung}

\subsection{Metriken}
Die Evaluierung der Ergebnisse der Segmentierung erfolgt auf Pixelebene.
\cite{long_fully_2015} berechnet 4 Metriken.
Sei \(n_{ij}\) die Anzahl der Pixel der Klasse \(i\) die der Klasse \(j\) zugeordnet wurden.

% \csvautotabular{results/DIVA-HisDB.csv}

\newcommand{\resulttable}[2]{
    \begin{tabular}{l|r|r|r|r|r}%
        \hline
        \csvreader[head to column names, filter equal={\dataset}{#1}]{results/document_image_segmentation_results.csv}{}%
        {#2}
        \\\hline
        \end
        {tabular}
}

\resulttable{CB55}{\\ \name & \pixelacc & \FgPA & \meanacc & \meanIU & \fwIU}
\resulttable{CSG18}{\\  \pixelacc & \FgPA & \meanacc & \meanIU & \fwIU}
\resulttable{CSG863}{\\  \pixelacc & \FgPA & \meanacc & \meanIU & \fwIU}
