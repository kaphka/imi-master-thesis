\chapter{Umsetzung}
\qq{Nebenziel bei der Umsetzung}
Ziel der Arbeit war es nicht nur Ergebnisse zu reproduzieren, sondern auch selbst reproduzierbar zu sein. 
Dafür sollten alle Prozesse, Datenverarbeitungsschritte und gewählte Paramter in einheitlicher Form definiert sein.  
\subsection{Python und PyTorch}
Python ist eine objektorientierte, interpretierte Programmiersprache.
Zur Bildverarbeitung wurde die Python-Bibliothek scikitimage verwendet.
Die neuronalen Netze wurden mithilfe von PyTorch umgesetzt.
% Genauer klären
PyTorch ist ein Framework zur automatischen Differenzierung von skalarer Funktion \autocite{PaszkeAutomaticdifferentiationPyTorch2017}.
\label{chap:umsetzung}
\section{Chen}
SLIC Paramter nicht spezifiziert 
aslic?
Netz Lernt, aber nur MNIST
\sfigure{Fehlerrate CheNet(4,4,0) während der Trainingsiterationen}{figures/plot/run_ChenNet_MNIST_log_ChenNet4_4_01522892733-tag-training_error.pdf}
Boundary pixel werden als GT label verwendet

\section{Xu}
multilabel vorhersage?

\section{Discrimitve}

\section{Evaluierung}
\label{chap:eval}


\sfigure{Beispiel aus dem DIBCO2013-Dateset}{figures/tasks/DIBCO2013-dataset.pdf}


\subsection{Metriken}
Die Evaluierung der Ergebnisse der Segmentierung erfolgt auf Pixelebene.
\cite{LongFullyconvolutionalnetworks2015} berechnet 4 Metriken.
Sei \(n_{ij}\) die Anzahl der Pixel der Klasse \(i\) die der Klasse \(j\) zugeordnet wurden.


\newcommand{\resulttable}[3]{
    \begin{tabular}{l|r|r|r|r|r}%
    \hline
        \csvreader[head to column names, filter equal={\dataset}{#2}]{#1}{}%
        {#3}
        \end{tabular}
}
\begin{table*}
    \resulttable{results/document_image_segmentation_results.csv}{CB55}{ \name & \pixelacc & \FgPA & \meanacc & \meanIU & \fwIU\\}
    \resulttable{results/document_image_segmentation_results.csv}{CSG18}{  \pixelacc & \FgPA & \meanacc & \meanIU & \fwIU\\}
    \resulttable{results/document_image_segmentation_results.csv}{CSG863}{  \pixelacc & \FgPA & \meanacc & \meanIU & \fwIU\\}
        
\end{table*}