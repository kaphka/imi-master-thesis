\chapter{Reproduktion bisheriger Ergebnisse}
\label{chap:reproduktion}
\epigraph{What I cannot create, I do not understand}{--- Richarch Feyman}
\qq{Warum NN?}
Künstliche Neuronale Netzwerke (kurz NN) dominieren den Forschungsbereich der 
Bilderkennung dank vieler Erfolge. 
Die Wissenschaft der NN ist ein Feld, dass durch die aktuelle Praxis mehr als durch Theorie geprägt ist. 

Die Reproduktion ist ein wichtiger Bestandteil in jedem Forschungsgebiet. 


Das erste Experiment basiert auf dem neusten Paper von Mitgliedern der DIVA-Gruppe \citeauthor*{ChenConvolutionalNeuralNetworks2017}. 

Das zweite Experiment basiert auf der Forschung der Gewinner \citeauthor*{XuPageSegmentationHistorical2017} 

\section{\cite{ChenConvolutionalNeuralNetworks2017}}

\subsection{Vorverarbeitung}
\citeauthor{ChenConvolutionalNeuralNetworks2017} nutzen den Superpixelalgorithmus SLIC (simple linerar iterative clustering)
\cite{AchantaSLICSuperpixels2010} um die Dokumentenseiten in Superpixel einzuteilen.
Ein CNN klassifiziert dann Superpixel statt jeden Pixel einzeln.
Während des Trainings wird das Ground-truth Label des Zentrumpixels als Label für den Superpixel verwendet.

\citeauthor{ChenConvolutionalNeuralNetworks2017} verweisen für die Details zur Vorverarbeitung 
auf ihr Paper \cite{ChenPageSegmentationHistorical2016}. 
In diesem Paper werden die Bilder vor Anwendung des SLIC-Algorithmus mit einem Faktor von \(2^{-3}\)) skaliert.



\section{Bildverarbeitung mittels Neuronaler Netzwerke}
\section{CNN}
\section{SLIC Superpixel}
\cite{AchantaSLICSuperpixels2010}


\subsection{PyTorch}
% Genauer klären
PyTorch ist ein Framework zur automatischen Differenzierung von skaler Funktion \autocite{PaszkeAutomaticdifferentiationPyTorch2017}.

\section{Dokumentensegmetierung mittels CNN}



\section{\textcite{XuPageSegmentationHistorical2017}}
\section{VGG}
\section{Deconvolution}
\section{Ergebnisse}