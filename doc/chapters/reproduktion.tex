\chapter{Reproduktion bisheriger Ergebnisse}
\label{chap:reproduktion}
\epigraph{What I cannot create, I do not understand}{--- Richarch Feyman}

Künstliche Neuronale Netzwerke (kurz NN) dominieren den Forschungsbereich der 
Bilderkennung dank vieler Erfolge. \qq{Warum NN?}
Die Wissenschaft der NN ist ein Feld, dass durch die aktuelle Praxis mehr als durch Theorie geprägt ist. 

Die Reproduktion ist ein wichtiger Bestandteil in jedem Forschungsgebiet. 


Das erste Experiment basiert auf dem neusten Paper von Mitgliedern der DIVA-Gruppe: \citeauthor*{ChenConvolutionalNeuralNetworks2017}. 
\qq{welche Experimente?}
Das zweite Experiment basiert auf der Forschung des Gewinners des IDCAR2017-Wettbewerbs: \citeauthor*{XuPageSegmentationHistorical2017} 

\section{Bildverarbeitung mittels Neuronaler Netzwerke}
Einleitung \cite{LeCunDeeplearning2015}
\section{Aktivierungsfunktionen}
\subsection{ReLU}
\begin{align}
    f\left(x\right) &= \text{max}(0,x)
\end{align}
\subsection{Softmax}

\section{SGD}
\section{CNN}
\qq{Filter als vorläufer von CNN}
CNN kombinieren zwei Konzepte der Bilderverarbeitung: Neuronale Netzwerke und Filter.
Klassifikationsprobleme wurde traditionell in zwei Schritten gelöst. Zuerst wurden 
Featuredeskriptoren entwickelt welche dann als Input für trainierbare Klassifizierer 
verwendet wurden \autocite[2353]{RawatDeepConvolutionalNeural2017}.

\qq{Filterbeispiel?}
\qq{}

\section{\cite{ChenConvolutionalNeuralNetworks2017}}

\subsection{Vorverarbeitung}
Eine der ersten Probleme bei der Verarbeitung von Dokumentenseiten ist ihre Größe.
Die Bilder im HisDB-Datenset sind mit einer Auflösung von \(4872 \times 6496\) wesentlich größer als andere Datensets.\qq{Welche?}

\citeauthor{ChenConvolutionalNeuralNetworks2017} skalieren deshalb alle Bilder mit einem Faktor von  \(2^{-3}\) und wenden dann den Superpixelalgorithmus SLIC (simple linerar iterative clustering) an
\autocite{AchantaSLICSuperpixels2010} um die Dokumentenseiten in Superpixel einzuteilen.
Ein \(28 \times 28\)-Bereich um das Zentrum des Superpixel wird dann mithilfe eines faltenden Neuronalen Netzwerks (engl Convolutional Neural Netzwerke, kurz CNN)
klassifiziert. Diese Klassifizierung wird dann allen Pixel innerhalb des Superpixels zugewiesen.

Während des Trainings wird das Ground-truth Label des Zentrumpixels als Label für den Superpixel verwendet.

\section{SLIC Superpixel}
Der SLIC-Algorithmus basiert auf dem k-Means-Algorithmus und teilt Pixel inhalb eines 5D-Raums in Cluster ein \autocite{AchantaSLICSuperpixels2010}. 
In jedem Arbeitsschritt werden Pixel dem Clusterzentrum mit der geringsten Distanz zugeordnet und danach werden die Clusterzentren neu berechnet.
Das Distanzmaß \(D_s\) zu den Clusterzentren \(k=[1,K]\) basiert auf den Farbabstand im Lab-Farbraum \(d_{lab}\) und den räumlichen Abstand \(d_{xy}\):

\begin{align}
    d_{lab} &= \sqrt{ \left( l_k - l_i \right)^2 + \left( a_k - a_i \right)^2 + \left( b_k - b_i \right)^2 }\\
    d_{xy}  &= \sqrt{ \left( x_k - x_i \right)^2 + \left(y_k - y_i \right)^2 }\\
    D_{s}   &= d_{lab} + \frac{m}{S} d_{xy}
\end{align}

Der Faktor \(m\) ermöglicht eine Gewichtung den zwei Distanzmaßen. Je höher
der Faktor desto kompakter werden die Superpixel. \cref{fig:slic_parameters}
zeigt das Ergebniss des Algorithmus mit unterschiedlichen Parameter  \(m\)
angewendet auf eine Dokumentenseite.

\begin{figure}
    \centering
    \caption{Die SLIC-Pixelgrenzen sind in rot dargestellt }
    \label{fig:slic_parameters}
    \subfloat[ \(m = 0.1\)]{
        \includegraphics[width=0.24\textwidth]{figures/img/mark_boundaries_m0.jpg}
    }
    \subfloat[\(m = 1\)]{
        \includegraphics[width=0.24\textwidth]{figures/img/mark_boundaries_m1.jpg}
    }
    \subfloat[\(m = 10\)]{
        \includegraphics[width=0.24\textwidth]{figures/img/mark_boundaries_m10.jpg}
    }
    \subfloat[\(m = 100\)]{
        \includegraphics[width=0.24\textwidth]{figures/img/mark_boundaries_m100.jpg}
    }
\end{figure}

\citeauthor{AchantaSLICSuperpixelsCompared2012} stellen später zwei wichtige Erweiterungen vor:
Normalisierung des Distanzmaßes und Adaptive-SLIC.

\citeauthor{ChenConvolutionalNeuralNetworks2017a} nennen keine Details zur Wahl
der Superpixel-Parameter.

\section{Xavier initialization}


\section{Dropout}

\subsection{PyTorch}
% Genauer klären
PyTorch ist ein Framework zur automatischen Differenzierung von skaler Funktion \autocite{PaszkeAutomaticdifferentiationPyTorch2017}.

\section{Training}


\section{\textcite{XuPageSegmentationHistorical2017}}
\section{VGG}
\section{Deconvolution}
\section{Ergebnisse}